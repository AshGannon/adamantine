\documentclass[letterpaper]{article}
\usepackage{amsmath}
\usepackage{array}
\usepackage{color}
\usepackage{graphicx}
\usepackage{float} % utiliser H pour forcer a mettre l'image ou on veut
\usepackage{lscape} % utilisation du mode paysage
\usepackage{mathbbol} % permet d'avoir le vrai symbol pour les reels grace a mathbb
\usepackage{enumerate} % permet d'utiliser enumerate
\usepackage{moreverb} % permet d'utiliser verbatimtab : conservation la tabulation
\usepackage{stmaryrd} % permet d'utiliser \llbrackedt et \rrbracket : double crochet
\usepackage[noabbrev]{cleveref} % permet d'utiliser cref and Cref
\usepackage{caption} % permet d'utiliser subcaption
\usepackage{subcaption} % permet d'utiliser subfigure, subtable, etc
\usepackage[margin=1.in]{geometry} % controle les marges du document


\newcommand\bn{\boldsymbol{\nabla}}
\newcommand\bo{\boldsymbol{\Omega}}
\newcommand\br{\mathbf{r}}
\newcommand\la{\left\langle}
\newcommand\ra{\right\rangle}
\newcommand\bs{\boldsymbol}
\newcommand\red{\textcolor{red}}
\newcommand\ldb{\{\!\!\{}
\newcommand\rdb{\}\!\!\}}
\newcommand\llb{\llbracket}
\newcommand\rrb{\rrbracket}

\renewcommand{\(}{\left(}
\renewcommand{\)}{\right)}
\renewcommand{\[}{\left[}
\renewcommand{\]}{\right]}


\begin{document}
\title{Adamantine}
\author{Bruno Turcksin} 
\date{}
\maketitle

\section{Introduction}
Adamantine is software design to simulate additive manufacturing. It is based on
deal.II and p4est. The first goal is to accurately represent the heat transfer
and the phase transition between the powder, the solid, and the liquid.

\section{Governing equations}
\subsection{Assumptions}
\begin{itemize}
  \item No movement in the liquid (Marangoni effect).
  \item No the evaporation of the material.
  \item No change of volume when the material changes phase.
  \item No loss of heat by radiative transfer.
  \item Material properties are constant per cell.
  \item We assume that the there is always a mushy zone (no isothermal change
    of phase).
\end{itemize}
\subsection{Heat equation}
\subsubsection{Weak form}
The heat equation without phase change is given by:
\begin{equation}
  \rho(T) C_p(T) \frac{\partial T}{\partial t} - \bn \cdot \(k\bn T\) = Q,
\end{equation}
where $\rho$ is the mass density, $C_p$ is the specific heat, $T$, is the
temperature, $k$ is the thermal conductivity, and $Q$ is the volumetric heat
source.

When there is a phase change, the heat equation is usually written in term of
the enthalpy, $h$:
\begin{equation}
  \frac{\partial h(T)}{\partial t} -  \bn \cdot \(k\bn T\) = Q.
  \label{enthalpy}
\end{equation}
In the absence of phase change, we have:
\begin{equation}
  h(T) = \int_{T_0}^T \rho(T) C_p(T) dT.
\end{equation}
Under the assumption of a phase change with a mushy zone, $C_p$ and $\rho$ are independent
of the temperature, we write:
\begin{equation}
  h(T) =      
  \begin{cases}
   \rho_s C_{p,s} T & \text{if } T<T_{s}\\
   \rho_s C_{p,s} T_s + \(\frac{\rho_s C_{p,s}+\rho_l C_{p,l}}{2} +
    \frac{\rho_s+\rho_l}{2}  \frac{\mathcal{L}}{T_l-T_s}\) (T-T_s) & \text{if } T>T_{s} \text{ and } T<T_l \\
    \rho_s C_{p,s} T_s + \frac{C_{p,s}+C_{p,l}}{2} (T_l - T_s) +
    \frac{\rho_s+\rho_l}{2} \mathcal{L} + \rho_s C_{p,l}
    (T-T_l) & \text{if } T>T_l.
  \end{cases}
\end{equation}
Since we only care about $\frac{\partial h{T}}{\partial t}$, we have:
\begin{equation}
  \frac{\partial h(T)}{\partial t} = 
  \begin{cases}
    \rho_s C_{p,s} \frac{\partial T}{\partial t} &  \text{if } T \leq T_{s}\\
     \(\rho_{\text{eff}} C_{p,\text{eff}} + \rho_{\text{eff}} \frac{\mathcal{L}}{T_l-T_s}\)
     \frac{\partial T}{\partial t}  & \text{if } T>T_{s} \text{ and } T<T_l \\
    \rho_l C_{p,l} \frac{\partial T}{\partial t} &  \text{if } T \geq T_{l}\\
  \end{cases}
\end{equation}
Note that we have a more complicated setup because we have two solid phase
(solid and powder). 

So far we haven't discussed $k$. $k$ is simply given by:
\begin{equation}
  k = 
  \begin{cases}
    k_s & \text{if } T \leq T_s \\
    k_s + \frac{k_l - k_s}{T_l - T_s} (T- T_s) & \text{if} T>T_s \text{ and }
    T<T_l \\
    k_l & \text{if } T \geq T_l
  \end{cases}
\end{equation}

Finally we can rewrite \Cref{enthalpy}, as a set of three equations:
\begin{itemize}
  \item if $T \leq T_s$, we have:
    \begin{equation}
      \frac{\partial T}{\partial t} = \frac{1}{\rho_s C_{p,s}} \(\bn \cdot \(k
      \bn T\) + Q\)
    \end{equation}
  \item if $T_s < T < T_l$, we have:
    \begin{equation}
      \frac{\partial T}{\partial t} = \frac{1}{\(\rho_{\text{eff}}
      C_{p,\text{eff}} + \rho_{\text{eff}} \frac{\mathcal{L}}{T_l-T_s}\)} \(
      \bn \cdot \(k \bn T\) + Q \)
    \end{equation}
  \item if $T \geq T$, we have:
    \begin{equation}
      \frac{\partial T}{\partial t} = \frac{1}{\rho_l C_{p,l}} \(\bn \cdot \(k
      \bn T\) + Q\)
    \end{equation}
\end{itemize}

Next, we will focus on the weak form of:
\begin{equation}
  \frac{\partial T}{\partial t} = \frac{1}{\rho C_{p}} \(\bn \cdot \(k
  \bn T\) + Q\).
\end{equation}
We have succesively with $\alpha = \frac{1}{\rho C_{p}}$:
\begin{equation}
 \int b_i \frac{\partial T_i b_j}{\partial t} = \int b_i \alpha \(\bn \cdot \(k
  \bn T_j b_j\) + Q\),
\end{equation}
\begin{equation}
  \int b_i b_j \frac{d T_j}{dt} = \int \alpha T_j b_i \bn \cdot \(k \bn b_j\) +
  \int b_i Q,
\end{equation}
\begin{equation}
  \(\int b_i b_j\) \frac{d T_j}{dt} = - \int \alpha T_j \bn b_i \cdot \(k \bn b_j\) +
  \int_{\partial} \alpha T_j b_i \boldsymbol{n}\cdot \(k \bn b_j\) + \int b_i Q.
\end{equation}

\subsubsection{Boundary Condition}
We are now interested in the boundary term  $\int_{\partial} \alpha T_j b_i
\boldsymbol{n}\cdot \(k \bn b_j\)$, in the interest of understanding the
physical meaning of this term, we will write it as:
\begin{equation}
 \int_{\partial} \alpha b_i \boldsymbol{n}\cdot \(k \bn T\)
  \label{boundary}
\end{equation}
If \Cref{boundary} is equal to zero, this means that $\bn T=0$. Physically this
condition correspond to a reflective boundary condition. In practise, we are
interested in two kind of boundary conditions: radiative loss and convection. In
practise, we are interested in two kind of boundary conditions: radiative loss
and convection.
\paragraph{Radiative Loss}
The Stefan-Boltzmann law describes the heat flux due to radiation as:
\begin{equation}
  -\boldsymbol{n} \cdot  \(k \bn T\) = \varepsilon \sigma \(T^4 -T_{\infty}^4\),
\end{equation}
with $\varepsilon$ the emissitivity and $\sigma$ the Stefan-Boltzmann constant.
The value of $\sigma$ is (from NIST):
\begin{equation}
  \sigma = 5.670374419 \times 10^{-8} \frac{W}{m^2 k^4}.
\end{equation}
We can write:
\begin{equation}
  \begin{split}
    \int_{\partial} \alpha b_i \boldsymbol{n} \cdot \(k\bn T\) &= 
    -\int_{\partial} \alpha b_i \varepsilon \sigma \(T^4 - T_{\infty}^4\),\\
    &= -\int_{\partial} \alpha b_i \varepsilon \sigma T^4 + 
    \int_{\partial} \alpha b_i \varepsilon T_{\infty}^4
  \end{split}
  \label{radiation}
\end{equation}
We can now use \Cref{radiation} to impose the radiative loss. However,
\Cref{radiation} is nonlinear. Thus, we need to use a Newton solver to impose
the boundary condition. This is less than ideal. Instead, we will linearize the
Stefan-Boltzmann equation:
\begin{equation}
  -\boldsymbol{n} \cdot \(k\bn T\) = h_{\text{rad}}\(T-T_{\infty}\),
\end{equation}
with
\begin{equation}
  h_{\text{rad}} = \varepsilon \sigma\(T+T_{\infty}\)\(T^2 + T_{\infty}^2\).
\end{equation}
Thus, we have:
\begin{equation}
  \begin{split}
    \int_{\partial} \alpha b_i \boldsymbol{n} \cdot \(k \bn T\) &= 
    -\int_{\partial} \alpha b_i h_{\text{rad}} \(T-T_{\infty}\),\\
    &=-\int_{\partial} \alpha h_{\text{rad}} \sum_j T_j b_i b_j +
    \int_{\partial} \alpha h_{\text{rad}} T_{\infty} b_i.
  \end{split}
\end{equation}


\section{Algorithmic choice}
\subsection{Matrix-free implementation}
The implementation is done matrix-free for the following reasons:
\begin{itemize}
  \item New architecture have little memory per core and so not having to store
    the memory is very interesting.
  \item Because the latency of the memory, a very important part of our problem
    is memory bound. It is therefore interesting to decrease memory access even
    at the cost of more computation.
  \item Because we have time-dependent nonlinear problem, we would need to
    rebuild the matrix at least every time step. Since the assembly needs to be
    redone so often, storing the matrix is not advantageous.
\end{itemize}

\subsection{Adaptive mesh refinement}
Usually, the powder layer is about 50 microns thick but the piece that is being
built is several centimeters long. Moreover, since the material is melted using
an electron beam or a laser, the melting zone is very localized. This means that
a uniform would require a very large number of cells in place where nothing
happens (material not heated yet or already cooled). Using AMR, we can refine
the zones that are of interest for during a given time.

\end{document}
