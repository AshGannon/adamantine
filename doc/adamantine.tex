\documentclass[letterpaper]{article}
\usepackage{amsmath}
\usepackage{array}
\usepackage{color}
\usepackage{graphicx}
\usepackage{float} % utiliser H pour forcer a mettre l'image ou on veut
\usepackage{lscape} % utilisation du mode paysage
\usepackage{mathbbol} % permet d'avoir le vrai symbol pour les reels grace a mathbb
\usepackage{enumerate} % permet d'utiliser enumerate
\usepackage{moreverb} % permet d'utiliser verbatimtab : conservation la tabulation
\usepackage{stmaryrd} % permet d'utiliser \llbrackedt et \rrbracket : double crochet
\usepackage[noabbrev]{cleveref} % permet d'utiliser cref and Cref
\usepackage{caption} % permet d'utiliser subcaption
\usepackage{subcaption} % permet d'utiliser subfigure, subtable, etc
\usepackage[margin=1.in]{geometry} % controle les marges du document


\newcommand\bn{\boldsymbol{\nabla}}
\newcommand\bo{\boldsymbol{\Omega}}
\newcommand\br{\mathbf{r}}
\newcommand\la{\left\langle}
\newcommand\ra{\right\rangle}
\newcommand\bs{\boldsymbol}
\newcommand\red{\textcolor{red}}
\newcommand\ldb{\{\!\!\{}
\newcommand\rdb{\}\!\!\}}
\newcommand\llb{\llbracket}
\newcommand\rrb{\rrbracket}

\renewcommand{\(}{\left(}
\renewcommand{\)}{\right)}
\renewcommand{\[}{\left[}
\renewcommand{\]}{\right]}


\begin{document}
\title{Adamantine}
\author{Bruno Turcksin} 
\date{}
\maketitle

\section{Introduction}
Adamantine is software design to simulate additive manufacturing. It is based on
deal.II and p4est. The first goal is to accurately represent the heat transfer
and the phase transition between the powder, the solid, and the liquid.

\section{Governing equations}
\subsection{Assumptions}
\begin{itemize}
  \item No movement in the liquid (Marangoni effect).
  \item No the evaporation of the material.
  \item No change of volume when the material changes phase.
  \item No loss of heat by radiative transfer.
  \item Material properties are constant per cell.
  \item If the change of phase is isothermal, i.e., no mushy zone, the enthalpy
    is discontinuous and belongs to $L^2$. Thus we should use discontinuous
    finite element. However, the temperature is always continuous and belongs to
    $H^1$. We assume that the there is always a mushy zone, so that we can use
    continuous finite element.
\end{itemize}
\subsection{Heat equation}
The heat equation without phase change is given by:
\begin{equation}
  \rho(T) C_p(T) \frac{\partial T}{\partial t} - \bn \cdot \(k\bn T\) = Q,
\end{equation}
where $\rho$ is the mass density, $C_p$ is the specific heat, $T$, is the
temperature, $k$ is the thermal conductivity, and $Q$ is the volumetric heat
source.

If there are phase changes, the heat equation becomes
\begin{equation}
  \rho(T) \frac{\partial h(T)}{\partial t} - \bn \cdot \(k\bn T\) = Q,
\end{equation}
where $h$ is the enthalpy per unit of mass.

\section{Algorithmic choice}
\subsection{Matrix-free implementation}
The implementation is done matrix-free for the following reasons:
\begin{itemize}
  \item New architecture have little memory per core and so not having to store
    the memory is very interesting.
  \item Because the latency of the memory, a very important part of our problem
    is memory bound. It is therefore interesting to decrease memory access even
    at the cost of more computation.
  \item Because we have time-dependent nonlinear problem, we would need to
    rebuild the matrix at least every time step. Since the assembly needs to be
    redone so often, storing the matrix is not advantageous.
\end{itemize}

\subsection{Adaptive mesh refinement}
Usually, the powder layer is about 50 microns thick but the piece that is being
built is several centimeters long. Moreover, since the material is melted using
an electron beam or a laser, the melting zone is very localized. This means that
a uniform would require a very large number of cells in place where nothing
happens (material not heated yet or already cooled). Using AMR, we can refine
the zones that are of interest for during a given time.

\end{document}
